%!TEX T-program = xelatex
\documentclass{cv}

\begin{document}
\header{Rohith Rokkam}{
  B.S. Computer Science (Honors) and Mathematics \\
  Stony Brook University, Spring 2019, Summa Cum Laude (GPA: 3.92) \\
  rohithrokkam@yahoo.com; (516) 506-1196; github.com/rrokkam
}

\begin{cvsection}{Employment}
  \entryheader{Software Engineering SMTS - Salesforce / Tableau Software}{05/22 - Present}
  \entry{Software Engineering MTS - Salesforce / Tableau Software}{01/20 - 04/22}
  {
    \item Designed and implemented a rework of the Tableau error-asserting framework to replace frequently misused APIs with hygienic versions and sensible defaults
    \item Improved performance of the preflight CI pipeline using profiling tools, preventing hangs on large PRs and reducing time taken by over 50\% on small PRs
    \item Coordinated cross-team efforts to remove insecure dependencies from Tableau Prep's Gradle builds
    \item Added functionality to allow client teams' usage of Tableau Prep to be individually tracked
    \item Wrote a technical specification for and prototyped a tool that detects and triages SLA-impacting defects for Tableau services using httpd logs
    \item Set up Snowflake data sources using ETLs written in Apache Flink and Kafka and created Tableau visualizations using those data sources to prioritize improvements to service observability
    \item As scrum lead, identified, triaged, and prioritized bugs causing customer-facing errors to drive availability of VizQL Server in Tableau Online from 99.5\% to 99.9\%
    \item Unified VizQL Server health checks, removing a class of high-visibility defects in Tableau Public by enabling the server to consistently self-heal
    \item Contributed features for Tableau's telemetry/resource tracing library aimed at helping developers debug cross-language performance issues
    \item Identified, triaged, debugged, and fixed critical shipblocking production defects using Splunk, Tableau, and New Relic
  }
  \entry{Research and Development Intern - Sandia National Laboratories}{06/18 - 08/19}
  {
    \item Implemented a parallelization layer for a C++ branch-and-bound based constrained-optimization solver framework.
    \item Collaborated with researchers to design and implement new features at supercomputing scale
    \item Wrote dynamic MPI code for high-performance computing scenarios.
  }
  \entry{Teaching Assistant - Theory of Computation}{Spring 2018/19}
  {
    \item Wrote and graded homework and exams on automata, languages, Turing machines, and complexity
    \item Lectured the class as a substitute and held regular office hours
  }
  \entry{Teaching Assistant -- Foundations of Computer Science}{Spring 2017}
  {
    \item Taught a 25-person weekly recitation section discrete math, formal logic, and proofs
  }
\end{cvsection}

\begin{cvsection}{Selected Projects}
  \entry{Peer-to-peer Filesystem}{}
  {
    \item Wrote an Airdrop-like P2P network in Python using FUSE, hosted on a multithreaded bootstrap server
  }
  \entry{Packet Sniffer}{}
  {
    \item Created a packet sniffer using raw sockets in Python that parsed TCP, UDP, IP, Ethernet, and DNS
  }
  \entry{Dynamic Memory Allocator}{}
  {
    \item Wrote a memory allocation library in C using a segmented free-list and optimizations from glibc malloc
  }
  \entry{Bash-like Shell}{}
  {
    \item Made a shell in C that supports output redirection, piping, signal handling, and background job support
  }
\end{cvsection}

\begin{cvsection}{Other Education}
  \entry{SBU Algorithms Lab}{}
  {
    \item Discussed algorithms, discrete math, and data structures (ex: B-trees, Bloom filters, DFT)
  }
  \entry{Selected Coursework}{}
  {
  \item Personal: Crafting Interpreters, Category Theory for Programmers, Information Theory
  \item Graduate: Algorithms (audited Master's and Ph.D. sections), Probability Theory, Algebra
  \item Undergraduate: Operating Systems, Linear Algebra, Network Programming
  }
\end{cvsection}
\end{document}
