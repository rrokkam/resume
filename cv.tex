%!TEX T-program = xelatex
\documentclass{cv}

\begin{document}
\header{Rohith Rokkam}{
  rohithrokkam@yahoo.com; (516) 506-1196; github.com/rrokkam
}

\begin{cvsection}{Employment}
  \entryheader{Senior Software Engineer, Tableau Prep Engine}{May '22 - Present}
  \entry{
    \item Building an open-source C$\#$ log pipeline and visualizations for analysis of Prep log files in Tableau
  }
  \entryheader{Software Engineer, Tableau Prep Engine}{May '21 - Apr '22}
  \entry{
    \item Implemented resource governance for embedding of Tableau Prep embeddings in external products
    \item Solved internal compiler errors leading to generation of incorrect queries
    \item Drove cross-team efforts to improve Java dependency handling in Prep's Gradle builds
    \item Mentored engineers in product design, scrum, and tools including Git, Gradle, Splunk and debuggers
  }
  \entryheader{Software Engineer, VizQL Server}{Jan '20 - May '21}
  \entry{
    \item Redesigned the Tableau error-asserting APIs to include error codes, and converted common misuses into compile-time failures 
    \item Added cross-language Java/C++ telemetry/resource tracing for use in flamegraphs
    \item Profiled the Tableau preflight CI pipeline and improved performance by over 50\%
    \item Wrote ETLs using Apache Flink and Kafka to identify SLA-impacting bugs in SaaS enviroments in logs, and prioritized and triaged those bugs across the Analytics org to improve availability to 99.9\%
  }
  \entryheader{Research and Development Intern - Sandia National Laboratories}{Jun '18 - Aug '19}
  \entry{
    \item Implemented a parallelization layer for a C++ branch-and-bound based optimization framework
    \item Collaborated with researchers to design and implement new features at supercomputing scale
    \item Wrote dynamic MPI code for high-performance computing scenarios
  }
  \entryheader{Teaching Assistant - Theory of Computation}{Spring 2018/19}
  \entry{
    \item Wrote and graded homework and exams on automata, languages, Turing machines, and complexity
    \item Lectured the class as a substitute and held regular office hours
  }
  \entryheader{Teaching Assistant - Foundations of Computer Science}{Spring 2017}
  \entry{
    \item Taught a 25-person weekly recitation section discrete math, formal logic, and proofs
  }
\end{cvsection}
\begin{cvsection}{Education}
  \entry{
    \item Personal: Crafting Interpreters, Category Theory for Programmers, Information Theory
  }
  \entryheader{Stony Brook University, B.S.}{Spring 2019}
  \entry{
    \item Computer Science (Honors) and Mathematics, Summa Cum Laude
    \item Undergraduate Coursework: Operating Systems, Network Programming, Linear Algebra
    \item Graduate Coursework: Algorithms, Probability, Algebra
  }
  \entryheader{Stony Brook Algorithms Lab}{}
  \entry{
    \item Worked on unsolved problems in algorithms, discrete math, and data structures 
    \item Read and discussed research on ex: B$\epsilon$-trees, skip lists, and Bloom filters
  }
\end{cvsection}
\begin{cvsection}{Projects}
  \entryheader{Interpreter}{}
  \entry{
    \item Wrote an interpreter in Java for Lox, a programming language including typechecking, functions, objects, and control flow.
  }
  \entryheader{Peer-to-peer Filesystem}{}
  \entry{
    \item Built an Airdrop-like P2P network in Python using FUSE, hosted on a multithreaded bootstrap server
  }
  \entryheader{Dynamic Memory Allocator}{}
  \entry{
    \item Wrote a memory allocation library in C using a segmented free-list and optimizations from glibc malloc
  }
  \entryheader{Bash-like Shell}{}
  \entry
  {
    \item Made a shell in C that supports output redirection, piping, signal handling, and background job support
  }
\end{cvsection}

\end{document}
