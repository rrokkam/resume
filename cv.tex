%!TEX T-program = xelatex
\documentclass{cv}

\begin{document}
\header{Rohith Rokkam}{
  B.S. Computer Science (Honors) and Mathematics \\
  Stony Brook University, Spring 2019, Summa Cum Laude (GPA: 3.92) \\
  rohithrokkam@yahoo.com; (516)506-1196; github.com/rrokkam
}

\begin{cvsection}{Experience}
  \entry{Software Engineering MTS - Tableau (Salesforce)}{01/20 - Present}
  {
    \item Enabled embedding of Tableau Prep into Salesforce by allowing clients to track resource usage separately
    \item Coordinated and led cross-team efforts to remove insecure dependencies from the Tableau Prep dependency tree, and championed system improvements to simplify future security patching
    \item Designed and implemented a rework of the Tableau error-asserting framework to replace frequently misused APIs with hygienic versions and sensible defaults
    \item Improved performance of the preflight CI pipeline using profiling tools, preventing hangs on large PRs (100+ files) and reducing time taken by over 50\% on small PRs.
    \item Wrote design documents and technical specifications for an automation tool that automatically detects and triages SLA-impacting defects across the Analytics organization
    \item Helped client teams debug cross-language performance issues through contributions to Tableau's C++/Java telemetry library
    \item Set up Snowflake data sources using ETLs written in Apache Flink and Kafka and created Tableau visualizations using those data sources to drive improvements to service observability
    \item As scrum lead, identified, triaged, and prioritized bugs causing customer-facing errors to drive availability of VizQL Server in Tableau Online from 99.5\% to 99.9\%.
    \item Unified VizQL Server health checks, removing a class of high-visibility defects in Tableau Public by enabling the server to consistently self-heal
    \item Found and triaged key sources of customer dissatisfaction in Tableau Online using Tableau and Splunk visualizations tracking metrics including golden signals
    \item Identified and debugged critical shipblocking production issues using Splunk, Tableau, and New Relic
  }
  \entry{Research and Development Intern - Sandia National Laboratories}{06/18 - 08/19}
  {
    \item Enabled a client team to solve constrained-optimization problems of interest by adding a parallelization layer to a C++ branch-and-bound solver framework.
    \item Wrote dynamic MPI code suitable for use in high-performance computing scenarios.
  }
  \entry{Teaching Assistant - Theory of Computation}{Spring 2018/19}
  {
    \item Wrote and graded homework and exams on automata, languages, Turing machines, and complexity
    \item Lectured the class as a substitute and held regular office hours
  }
\end{cvsection}

\begin{cvsection}{Selected Projects}
  \entry{Peer-to-peer Filesystem}{}
  {
    \item Developed an Airdrop-like P2P service in Python using FUSE and a custom protocol
    \item Made a multithreaded bootstrap server to host the network
  }
  \entry{Packet Sniffer}{}
  {
    \item Created a packet sniffer using raw network sockets in Python
    \item Added output filters for protocols including TCP, UDP, IP, Ethernet, and DNS
  }
  \entry{Dynamic Memory Allocator}{}
  {
    \item Wrote a memory allocation library in C using a segmented free-list and optimizations from glibc malloc
  }
  \entry{Bash-like Shell}{}
  {
    \item Made a shell in C with output redirection, piping, and background job support
    \item Carefully considered race conditions and handled asynchronous UNIX signals
  }
\end{cvsection}

\begin{cvsection}{Organizations}
  \entry{SBU Algorithms Lab}{}
  {
    \item Discussed algorithms, discrete math, and data structures (ex: B-trees, Bloom filters, DFT)
  }
\end{cvsection}

\cvsectionheader{Selected Coursework}
\begin{minipage}{\textwidth}
  \begin{itemize}[noitemsep,topsep=0pt]%
    \item Graduate: Algorithms (audited Master's and Ph.D. sections), Probability Theory, Algebra
    \item Undergraduate: Operating Systems, Linear Algebra, Network Programming
    \item Personal: Category Theory for Programmers, Information Theory
  \end{itemize}%
\end{minipage}%
\end{document}
