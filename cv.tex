%!TEX T-program = xelatex
\documentclass{cv}

\begin{document}
\header{Rohith Rokkam}{
  B.S. Computer Science (Honors) and Mathematics \\
  Stony Brook University, Spring 2019, Summa Cum Laude (GPA: 3.92) \\
  rohithrokkam@yahoo.com; (516)5061196; github.com/rrokkam
}

\begin{cvsection}{Experience}
  \entry{Software Engineering MTS - Tableau (Salesforce)}{01/20 - Present}
  {
    \item Prometheus
    \item Log4J
    \item Hygiene: preflight performance, prep performance, service upgrades
    \item logicasserts and debugasserts
    \item Enabled client teams to debug cross-language performance issues through contributions to Tableau's C++/Java telemetry library
    \item Set up Snowflake data sources using ETLs written in Apache Flink and Kafka and created Tableau visualizations using those data sources to improve service observability
    \item Identified and triaged key sources of customer dissatisfaction in Tableau Online using Tableau and Splunk visualizations tracking golden signals
    \item As scrum lead, organized backlog and prioritized tasks to improve availability of VizQL Server in Tableau Online from 99.5\% to 99.9\%.
    \item Wrote design documents and technical specifications for automation to automatically detect and triage SLA-impacting defects across the Analytics organization
    \item Removed a class of high-visibility defects in Tableau Public by unifying VizQL Server health checks, enabling the server to consistently self-heal
    \item Identified and helped debug critical shipblocking production issues using Splunk, Tableau, and New Relic
  }
  \entry{Research and Development Intern - Sandia National Laboratories}{06/18 - 08/19}
  {
    \item Enabled a client team to solve constrained-optimization problems of interest by adding a parallelization layer for a C++ branch-and-bound solver framework.
    \item Wrote dynamic MPI code suitable for use in high-performance computing scenarios.
  }
  \entry{Teaching Assistant - Theory of Computation}{Spring 2018/19}
  {
    \item Wrote and graded homework and exams on automata, languages, Turing machines, and complexity
    \item Lectured the class as a substitute and held regular office hours
  }
\end{cvsection}

\begin{cvsection}{Selected Projects}
  \entry{Peer-to-peer Filesystem}{}
  {
    \item Wrote an Airdrop-like P2P service in Python using FUSE and a custom protocol
    \item Made a multithreaded bootstrap server to host the network
  }
  \entry{Packet Sniffer}{}
  {
    \item Created a packet sniffer using raw network sockets in Python
    \item Added output filters for protocols including TCP, UDP, IP, Ethernet, and DNS
  }
  \entry{Dynamic Memory Allocator}{}
  {
    \item Developed a memory allocation library in C using a segmented free-list
    \item Implemented several optimizations from glibc malloc
  }
  \entry{Bash-like Shell}{}
  {
    \item Made a shell in C with output redirection, piping, and background job support
    \item Carefully considered race conditions and handled asynchronous UNIX signals
  }
\end{cvsection}

\begin{cvsection}{Organizations}
  \entry{SBU Algorithms Lab}{}
  {
    \item Discussed algorithms, discrete math, and data structures (ex: B-trees, Bloom filters, DFT)
  }
\end{cvsection}

\cvsectionheader{Selected Coursework}
\begin{minipage}{\textwidth}
  \begin{itemize}[noitemsep,topsep=0pt]%
    \item Graduate: Algorithms (audited Master's and Ph.D. sections), Probability Theory, Algebra
    \item Undergraduate: Operating Systems, Linear Algebra, Network Programming
    \item Personal: Category Theory for Programmers, Information Theory
  \end{itemize}%
\end{minipage}%
\end{document}
