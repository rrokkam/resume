%!TEX T-program = xelatex
\documentclass{cv}

\begin{document}
\header{Rohith Rokkam}

\begin{aside}
  \section{education}
    Stony Brook University
    B.S. Comp. Sci., Math
    Honors College
    GPA: 3.94
    expected graduation 2019
  \section{contact}
    rohithrokkam@yahoo.com
    (516) 506-1196
    github.com/rrokkam
\end{aside}

\section{interests}

concurrent data structures, high-performance computing, algorithm design

\section{experience}

\begin{entrylist}
  \entry{2018}{Intern @ Sandia National Labs, Albuquerque, NM}
    {Contributed to parallelization facilities for PEBBL, a MPI-based C++ framework for solving branch-and-bound problems.}
\end{entrylist}

\section{projects}

\begin{entrylist}
  \entry{2018}{Peer-to-peer Filesystem}
    {A distributed filesystem written in Python using the FUSE library, using a bootstrap server to host.}
  \entry{2018}{Terminal Chat Service}
    {A terminal chat client in C and multithreaded server in Python. Sends messages using a custom plaintext protocol.}
  \entry{2018}{Packet Sniffer}
    {A packet sniffer implemented using Python's raw sockets. Dumps packets to human-readable, hex, or pcap (Wireshark-readable) forms.}
  \entry{2017}{Dynamic Memory Allocation Library}
    {A memory allocator written in C, using a segmented free-list and first-fit allocation policy.}
  \entry{2017}{Shell}
    {A shell written in C with bash-like output redirection and job control.}
  \entry{2017}{Memcached Clone}
    {An in-memory caching service in C, implemented using a custom multithreading-safe queue and hashmap.}
  \entry{2016}{Navigation System}
    {A GPS developed in Java using data from OpenStreetMap and an XML parser. Implements Djikstra's shortest-path algorithm to calculate routes.}
\end{entrylist}

\section{teaching}

\begin{entrylist}
  \entry{2018}{Teaching Assistant: Theory of Computation}
    {Wrote \& graded homework on finite automata, formal languages, and Turing machines. Also taught as a substitute.}
  \entry{2017}{Teaching Assistant: Foundations of Computer Science}
    {Instructed 20-person recitation section on discrete math, logic, and proof techniques.}
\end{entrylist}

\end{document}
