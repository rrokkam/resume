%!TEX T-program = xelatex
\documentclass{cv}

\begin{document}
\header{Rohith Rokkam}
\rule{\textwidth}{0.4pt}
\begin{center}
B.S. Computer Science (Honors) and Mathematics, Summa Cum Laude \\
Stony Brook University, Spring 2019

rohithrokkam@yahoo.com; (516)506-1196; github.com/rrokkam 

\end{center}

% Work these into the bullet lists below
%algorithms, discrete math, probability, high-performance computing, data structures, concurrency

\section{experience}

\begin{entrylist}
  \entry{06/18 - 06/19}{Research and Development Intern, Center for Computing Research, Sandia National Labs}
    {Contributed to parallelization facilities for PEBBL, a C++ framework for solving branch-and-bound problems. Wrote dynamic MPI code with a focus on minimizing communication overhead and maintaining legacy compatibility.}
    \entry{01/19 -}{Member, Stony Brook Algorithms Lab}
    {Discuss research topics in the theory of computer science. We read papers on topics related to the theory of computer science and present the topics to one another.}
\end{entrylist}

\section{projects}

\begin{entrylist}
  \entryy{Fall 2018}{Canvassing Application}
  {
    \item Item 1
    \item Item 2
    \item Item 3
  }
%  \entry{Fall 2018}{Canvassing Application}
%  {\vspace{-3.5mm}
%    \begin{itemize}[noitemsep,topsep=0pt]
%    \item Item 1
%    \item Item 2
%    \item Item 3
%    \end{itemize}
%  }
%    {\vspace{-\topsep}\begin{itemize}%
%      \item A web application written in JavaScript and Python that helps manage door-to-door campaigns. 
%      \item Keeps track of availabilities of canvassers and assigns houses to canvassers using Google's Vehicle Routing Problem solver. I worked mostly on the Python backend.
%    \end{itemize}}

%    {Wrote a web application in JavaScript and Python that helps manage door-to-door campaigns, as part of a group. Keeps track of availabilities of canvassers and assigns houses to canvassers using Google's Vehicle Routing Problem solver. Information stored using MongoDB. I worked mostly on the Python backend.}
  \entry{Spring 2018}{Peer-to-peer Filesystem}
    {Written using the FUSE bindings for Python. The P2P network is hosted by a multithreaded bootstrap server and connects using a custom protocol. Functionality similar to Airdrop. Mountable on Linux and MacOS.}
  \entry{Spring 2018}{Packet Sniffer}
    {Implemented using raw sockets in Python. Dumps packets to human-readable, hex, or pcapng (Wireshark-readable) formats as desired. Optionally filters packets by protocol.}
  \entry{Fall 2017}{Dynamic Memory Allocator}
    {Developed in C, using a first-fit allocation policy. Stores free blocks with a variable-size segmented free-list. Implements several optimizations found in glibc malloc, such as use of a wilderness block.}
  \entry{Fall 2017}{Shell}
    {A shell written in C with bash-like features, including output redirection, piping, and background jobs. Carefully implements UNIX signal handling and manages the life cycle of spawned processes.}
  \entry{Fall 2016}{Navigation System}
    {A Google Maps-like application developed in Java using the OpenStreetMap API and an XML parser, using a custom implementation of Djikstra's shortest-path algorithm for route computation.}
\end{entrylist}

\section{teaching}

\begin{entrylist}
  \entry{Spring 2018/19}{Teaching Assistant: Theory of Computation}
    {Wrote \& graded homework and exams on finite automata, formal languages, Turing machines, and complexity theory.}
  \entry{Spring 2017}{Teaching Assistant: Foundations of Computer Science}
    {Instructed 20-person recitation section on discrete math, logic, and proof techniques.}
\end{entrylist}

\section{personal}

\begin{entrylist}
  \entry{Fall 2017 - Spring 2019}{Stony Brook Go Club}
  {Secretary, dc trip, gotham
  }
  \entry{Fall 2017 - Spring 2018}{SBU Undergrad Algorithms Reading Group}
  {Present algorithms and data structures of interest.
  } 
\end{entrylist}

\end{document}
