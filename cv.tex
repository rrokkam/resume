%!TEX T-program = xelatex
\documentclass{cv}

\begin{document}
\header{Rohith Rokkam}

\begin{aside}
  \section{education}
    Stony Brook University
    B.S. Comp. Sci., Math
    Honors Computer Science
    GPA: 3.9
    graduation: spring 2019
  \section{contact}
    rohithrokkam@yahoo.com
    (516) 506-1196
    github.com/rrokkam
\end{aside}

\section{interests}

algorithm design, high-performance computing, concurrent data structures

\section{experience}

\begin{entrylist}
  \entry{2018-19}{Center for Computing Research, Sandia National Labs, Albuquerque, NM}
    {As an intern, Contributed to parallelization facilities for PEBBL, a MPI-based C++ framework for solving branch-and-bound problems. 
    Paper titled ``The Implementation of Parallel Bounding in PEBBL'' forthcoming in the Sandia Intern Proceedings.}
\end{entrylist}

\section{projects}

\begin{entrylist}
  \entry{2018}{Peer-to-peer Filesystem}
    {A distributed filesystem similar to Airdrop, written using the Python bindings for the FUSE library. The P2P network is hosted by a multithreaded bootstrap server. Mountable on Linux and MacOS.}
  \entry{2018}{Terminal Chat Service}
    {A terminal chat client in C and multithreaded server in Python. Messages are sent using a custom plaintext protocol.}
  \entry{2018}{Packet Sniffer}
    {A packet sniffer implemented using Python's raw sockets. Dumps packets to human-readable, hex, or pcapng (Wireshark-readable) formats as desired. Optionally filters packets by protocol.}
  \entry{2017}{Dynamic Memory Allocation Library}
    {A memory allocator developed in C, using a first-fit allocation policy. Stores free blocks with a variable-size segmented free-list. Borrows several optimizations from glibc's malloc, such as use of a wilderness block.}
  \entry{2017}{Shell}
    {A shell written in C with bash-like features, including output redirection, piping, and job control.}
  \entry{2017}{Memcached Clone}
    {An in-memory caching service in C, implemented using a custom multithreading-safe queue and hashmap.}
  \entry{2016}{Navigation System}
    {A Google Maps-like application developed in Java using the OpenStreetMap API and an XML parser. Implements Djikstra's shortest-path algorithm to compute shortest routes.}
\end{entrylist}

\section{teaching}

\begin{entrylist}
  \entry{2018-19}{Teaching Assistant: Theory of Computation}
    {Wrote \& graded homework on finite automata, formal languages, and Turing machines. Also taught as a substitute.}
  \entry{2017}{Teaching Assistant: Foundations of Computer Science}
    {Instructed 20-person recitation section on discrete math, logic, and proof techniques.}
\end{entrylist}

\end{document}
