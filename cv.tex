%!TEX T-program = xelatex
\documentclass{cv}

\begin{document}
\header{Rohith Rokkam}{
  B.S. Computer Science (Honors) and Mathematics, Summa Cum Laude \\
  Stony Brook University, Spring 2019

  rohithrokkam@yahoo.com; (516)506-1196; github.com/rrokkam
}

\begin{cvsection}{experience}
  \entry{06/18 - 06/19}{Research and Development Intern}{Sandia National Labs}
  {
    \item Contributed to parallelization facilities for PEBBL, a C++ branch-and-bound framework. CCR
    \item Wrote dynamic MPI code with a focus on minimizing communication overhead and maintaining legacy compatibility.
  }
   \entry{Spring 2018/19}{Teaching Assistant}{Theory of Computation}
   {
     \item Wrote \& graded homework and exams on finite automata, formal languages, Turing machines, and complexity theory.
   }
   \entry{Spring 2017}{Teaching Assistant}{Foundations of Computer Science}
   {
     \item Instructed 20-person recitation section on discrete math, logic, and proof techniques.
   }
\end{cvsection}

\begin{cvsection}{projects}
  \entry{Fall 2018}{Canvassing Application}{}
  {
    \item Collaborated to build a JavaScript web app for managing door-to-door campaigns.
    \item Made a microservice in Python using MongoDB and Google's OR-Tools.
  }
  \entry{Spring 2018}{Peer-to-peer Filesystem}{}
  {
    \item Wrote an Airdrop-like P2P service for Linux and MacOS using Python's FUSE bindings.
    \item Designed a custom protocol and multithreaded bootstrap server to host the network.
  }
  \entry{Spring 2018}{Packet Sniffer}{}
  {
    \item Implemented a packet sniffer using raw sockets in Python. 
    \item Writes packets in human-readable, hex, or pcapng (Wireshark-readable) formats, and can filter by protocol.
  }
  \entry{Fall 2017}{Dynamic Memory Allocator}{}
  {
    \item Developed a malloc library in C, using first-fit allocation with a segmented free-list.
    \item Implements some optimizations from glibc malloc, ex: wilderness block.
  }
  \entry{Fall 2017}{Shell}{}
  {
    \item Written in C with bash-like features and syntax, including output redirection, piping, and background jobs. 
    \item Carefully implements UNIX signal handling and process life-cycle management.
  }
  \entry{Fall 2016}{Navigation System}{}
  {
    \item Developed in Java using the OpenStreetMap API and an XML parser, with functionality similar to Google Maps.
    \item Wrote a custom implementation of Djikstra's shortest-path algorithm for directions.
  }
\end{cvsection}

\begin{cvsection}{organizations}
  \entry{01/19 -}{SBU Algorithms Lab}{}
  {
    \item Discuss research topics in the theory of computer science.
    \item We read papers on topics related to the theory of computer science and present the topics to one another.
    \item algorithms, discrete math, probability, high-performance computing, data structures, concurrency
  }
  \entry{Fall 2017 - Spr. 2019}{SBU Go Club}{}
  {
    \item secretary, dc trip, gotham
  }
  \entry{Fall 2017 - Spr. 2018}{SBU Undergrad Algorithms Reading Group}{}
  {
    \item Present algorithms and data structures of interest.
  } 
\end{cvsection}

\end{document}
